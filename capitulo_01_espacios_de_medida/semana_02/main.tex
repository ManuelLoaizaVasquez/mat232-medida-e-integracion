\documentclass{article}
\usepackage[utf8]{inputenc}
\usepackage{amsfonts,latexsym,amsthm,amssymb,amsmath,amscd,euscript}
\usepackage{mathtools}
\usepackage{framed}
% Descomentar fullpage cuando se quiera utilizar menos margen horizontal
%\usepackage{fullpage}
\usepackage{hyperref}
    \hypersetup{colorlinks=true,citecolor=blue,urlcolor =black,linkbordercolor={1 0 0}}

\newenvironment{statement}[1]{\smallskip\noindent\color[rgb]{1.00,0.00,0.50} {\bf #1.}}{}
\allowdisplaybreaks[1]

% Suelo y techo
\DeclarePairedDelimiter\ceil{\lceil}{\rceil}
\DeclarePairedDelimiter\floor{\lfloor}{\rfloor}

% Comandos para teoremas, definiciones, ejemplos, lemas, etc. para sus respectivos body types.
\renewcommand*{\proofname}{Prueba}
\renewcommand{\contentsname}{Contenido}

\newtheorem{theorem}{Teorema}
\newtheorem*{proposition}{Proposici\'on}
\newtheorem{lemma}[theorem]{Lema}
\newtheorem{corollary}[theorem]{Corolario}
\newtheorem{conjecture}[theorem]{Conjetura}
\newtheorem*{postulate}{Postulado}
\theoremstyle{definition}
\newtheorem{defn}[theorem]{Definici\'on}
\newtheorem{example}[theorem]{Ejemplo}

\theoremstyle{remark}
\newtheorem*{remark}{Observaci\'on}
\newtheorem*{notation}{Notaci\'on}
\newtheorem*{note}{Nota}

% Define tus comandos para hacer la vida más fácil.
\newcommand{\BR}{\mathbb R}
\newcommand{\BC}{\mathbb C}
\newcommand{\BF}{\mathbb F}
\newcommand{\BQ}{\mathbb Q}
\newcommand{\BZ}{\mathbb Z}
\newcommand{\BN}{\mathbb N}

\title{MAT232 Medida e Integraci\'on}
\author{Manuel Loaiza Vasquez}
\date{Abril 2021}

\begin{document}

\maketitle

\vspace*{-0.25in}
\centerline{Pontificia Universidad Cat\'olica del Per\'u}
\centerline{Lima, Per\'u}
\centerline{\href{mailto:manuel.loaiza@pucp.edu.pe}{{\tt manuel.loaiza@pucp.edu.pe}}}
\vspace*{0.15in}

\begin{framed}
  Solucionario de la Tarea 2 que abarca los temas de las Semanas 2 - 3 del curso
  Medida e Integraci\'on de la especialidad de Matem\'aticas de la Facultad de
  Ciencias e Ingenier\'ias dictado por el profesor Johel Beltr\'an
  durante el ciclo $2021-1$.
\end{framed}

\begin{defn}[Sigma Álgebra de Borel]
Sea $\mathcal{O}_m$ la colección formada por los abiertos de $\mathbb{R}^m$.
Definiremos como $\sigma$-álgebra de Borel al sigma algebra generado por
$\mathcal{O}_m$. Es decir,  $\mathcal{B}_m$ = $\sigma(\mathcal{O}_m)$.
Además a los elementos de $\mathcal{B}_m$ se les denominará Borelianos o Borel
medibles.
\end{defn}

\begin{statement}{1}
Denotemos por $\mathcal{K}_m$ a la colección de compactos de $\mathbb{R}^m$ y
por $\mathcal{C}_m$ a la colección de cerrados de $\mathbb{R}^m$.
\begin{enumerate}
  \item Pruebe que $\mathcal{B}_m$ = $\sigma(\mathcal{K}_m)$
  \item Pruebe que $\mathcal{B}_m$ = $\sigma(\mathcal{C}_m)$
\end{enumerate}
\end{statement}

\begin{proof}
Primero probemos el siguiente lema.
  \begin{lemma}
    Se cumple que $\sigma(\mathcal{K}_m) = \sigma(\mathcal{C}_m)$.
  \end{lemma}
  \begin{proof}
    Tenemos que $\mathcal{K}_m \subset \sigma(\mathcal{C}_m)$ trivialmente
    ya que todo compacto es cerrado, por lo que
    $\sigma(\mathcal{K}_m) \subset \sigma(\mathcal{C}_m)$.
    Ahora probemos que $\mathcal{C}_m \subset \sigma(\mathcal{K}_m)$.
    Sea $c \in C_m$, escribamos $c = c \cap \BR^m$. Asimismo, $(A_n)_{n \in \BZ}$
    con $A_n = [n, n + 1] \times \dots \times [n, n + 1]$ cumple que
    $\cup_{n \in \BZ} A_n = \BR^m$. De esta manera tenemos
    $c = c \cap \cup_{n \in \BZ} A_n = \cup_{n \in \BZ} (c \cap A_n)$.
    Debido a que la intersecci\'on de un conjunto cerrado con un conjunto cerrado
    y acotado es un conjunto cerrado y acotado, entonces
    $c \cap A_n$ es cerrado y acotado y $\cup_{n \in \BZ} (c \cap A_n)
    \in \mathcal{K}_m$. As\'i, conseguimos $\sigma(\mathcal{C}_m) \subset \sigma(\mathcal{K}_m)$.
    Ahora podemos concluir que $\sigma(\mathcal{C}_m) = \sigma(\mathcal{K}_m)$.
  \end{proof}
  Para poder rematar el teorema, probemos que
  $\sigma(\mathcal{O}_m) = \sigma(\mathcal{C}_m)$.
  Est\'a claro que $\mathcal{O}_m \subset \sigma(\mathcal{C}_m)$ pues dado
  un conjunto abierto $o \in \mathcal{O}_m$, $o^C$ es un conjunto cerrado.
  $o^C \in \sigma(\mathcal{C}_m)$ y por ser $\sigma$-\'algebra, tenemos que
  $(o^C)^C = o \in \sigma(\mathcal{C}_m)$.
  As\'i, $\sigma(\mathcal{O}_m) \subset \sigma(\mathcal{C}_m)$. Como el
  complemento de un conjunto cerrado es un conjunto abierto, entonces probar
  $\sigma(\mathcal{C}_m) \subset \sigma(\mathcal{O}_m)$ se realiza de manera
  an\'aloga.
  Finalmente podemos concluir que $\mathcal{B}_m = \sigma(\mathcal{O}_m) =
  \sigma(\mathcal{C}_m) = \sigma(\mathcal{K}_m)$.
\end{proof}

\begin{statement}{2}
  Denotemos por $\mathcal{J}_m^o$ a la colección de bloques abiertos de
  $\mathbb{R}^m$, por $\mathcal{J}_m$ a la de semiabiertos (producto de
  intervalos abiertos por derecha y cerrados por izquierda),
  por $\mathcal{J}_{m,q}^o$ a la de abiertos con extremos racionales;
  y por $\mathcal{J}_{m,q}$ a la de semiabiertos con extremos racionales.
  Pruebe que  $\mathcal{B}_m$ = $\sigma(\mathcal{J}_m^o)$ =
  $\sigma(\mathcal{J}_m)$ = $\sigma(\mathcal{J}_{m,q}^o)$ =
  $\sigma(\mathcal{J}_{m,q})$.
\end{statement}

\begin{proof}
  Primero probemos este lema:
  \begin{lemma}
    Se cumple que $\mathcal{B}_m = \sigma(\mathcal{J}_{m, q}^o)$.
  \end{lemma}
  \begin{proof}
    Probar que $\sigma(\mathcal{J}_{m, q}^o) \subset \mathcal{B}_m = \sigma(\mathcal{O}_m)$
    es sencillo puesto que todo intervalo abierto con extremos racionales es un
    conjunto abierto.
    ahora probemos que $\mathcal{B}_m = \sigma(\mathcal{O}) \subset \sigma(\mathcal{J}_{m, q}^o)$.
    Primero, definamos la caja $B_{q, r} = \prod_{i = 1}^m (q_i - r, q_i + r)$.
    Dado un $o \in \mathcal{O}_m$ podemos escribir este como
    $\cup_{x \in o} B_{x_q, r_q}$, uni\'on de bolas rectangulares con centro en
    $x_q$ y radio $r_q$. Este $x_q$ es un punto racional tal que $x$ pertenece
    a su bola y esta \'ultima est\'a contenida en $o$. Este punto $x_q$ y radio
    $r_q$ podemos conseguirlos siempre por la densidad de $\BQ_m$ en $\BR_m$.
    Luego, tenemos $\{B_{x_q, r_q} : x \in o\} \subset \{B_{q, r} : q \in \BQ^m \text{ y } r \in \BQ\}$.
    y este \'ultimo conjunto es numerable, por lo que el primer conjunto no le queda de otra
    que ser numerable. De esta manera, hemos escrito al conjunto $o$ como una uni\'on enumerable
    de conjuntos en $\sigma(\mathcal{J}_{m, q}^o)$, por lo que tendr\'iamos
    $\mathcal{B}_m = \sigma(\mathcal{O}) \subset \sigma(\mathcal{J}_{m, q}^o)$.
  \end{proof}
  Ahora probemos el siguiente lema:
  \begin{lemma}
    Se cumple que $\sigma(\mathcal{J}_{m, q}) = \sigma(\mathcal{J}_{m, q}^o)$.
  \end{lemma}
  \begin{proof}
    Probemos que $\sigma(\mathcal{J}_{m, q}) \subset \sigma(\mathcal{J}_{m, q}^o)$.
    Sea $j = \prod_{i = 1}^m [l_i, r_i) \in \mathcal{J}_{m, q}$,
    con $l_i, r_i \in \BQ$ y $1 \leq i \leq m$.
    Tenemos que $j$ puede ser escrito como una intersecci\'on enumerable de
    elementos en $\sigma(\mathcal{J}_{m, q}^o)$, pues
    \[
      \bigcap_{n \in \BN} \left(l_i - \frac{1}{n}, r_i\right) = [l_i, r_i)
    \]
    por lo que $\sigma(\mathcal{J}_{m, q}) \subset \sigma(\mathcal{J}_{m, q}^o)$.
    Ahora probemos la otra direcci\'on.
    Sea $j = \prod_{i = 1}^m (l_i, r_i) \in \mathcal{J}_{m, q}^o$,
    con $l_i, r_i \in \BQ$ y $1 \leq i \leq m$.
    Tenemos que $j$ puede ser escrito como una uni\'on enumerable de elementos
    en $\sigma(\mathcal{J}_{m, q})$, pues
    \[
      \bigcup_{n \in \BN} \left[l_i + \frac{1}{n + \floor*{\frac{1}{r_i - l_i}}}, r_i\right) = (l_i, r_i)
    \]
    por lo que $\sigma(\mathcal{J}_{m, q}^o) \subset \sigma(\mathcal{J}_{m, q})$.
    Finalmente, podemos concluir con $\sigma(\mathcal{J}_{m, q}) = \sigma(\mathcal{J}_{m, q}^o)$.
  \end{proof}
  De manera an\'aloga a este \'ultimo lema tenemos $\sigma(\mathcal{J}_m) = \sigma(\mathcal{J}_m^o)$.
  Finalmente. como
  \begin{gather*}
    \sigma(\mathcal{J}_{m, q}^o) \subset \sigma(\mathcal{J}_m^o) \subset \sigma(\mathcal{O})
    \subset \sigma(\mathcal{J}_{m, q}^o)\\
    \sigma(\mathcal{J}_{m, q}^o) = \sigma(\mathcal{J}_{m, q})\\
    \sigma(\mathcal{J}_m) = \sigma(\mathcal{J}_m^o)
  \end{gather*}
  entonces podemos concluir que
  \[
    \mathcal{B}_m = \sigma(\mathcal{J}_m) = \sigma(\mathcal{J}_m^o) =
    \sigma(\mathcal{J}_{m, q}^o) = \sigma(\mathcal{J}_{m, q}).
  \]
\end{proof}

\begin{statement}{3}
  Denote por $\mathcal{R} = \{B_1 \times B_2 : B_1, B_2 \in \mathcal{B}_1\}$.
  Pruebe lo siguiente:
  \begin{enumerate}
    \item Si $B_1, B_2 \in \mathcal{B}_1$, entonces $B_1 \times B_2 \in \mathcal{B}_2$.
    \item Concluya que $\sigma(\mathcal{R}) = \mathcal{B}_2$.
  \end{enumerate}
\end{statement}

\begin{proof}
\end{proof}

\begin{statement}{4}
  Sea $F: \BR \to \BR$ una funci\'on creciente y continua por derecha.
  Pruebe que la aplicaci\'on
  \[
    \mu_F((a, b]) = F(b) - F(a), \text{ para todo } a, b \in \BR \text{ tal que } a < b
  \]
  tiene una \'unica extensi\'on a medida $\nu_F$ sobre los borelianos.
\end{statement}

\begin{proof}
\end{proof}

\begin{defn}[Espacio de Medida Completo]
  Sea $(X, \mathcal{F}, \mu)$ un espacio de medida. Decimos que este es completo
  si todo subconjunto de un medible de medida cero es tambi\'en medible.
\end{defn}

\begin{statement}{5}
  Sea $(X, \mathcal{F}, \mu)$ un espacio de medida y sea
  \[
    N = \{E \subset X : \text{ existe } B \in \mathcal{F} \text{ con }
    E \subset B \text{ y } \mu(B) = 0\}.
  \]
  Pruebe lo siguiente:
  \begin{enumerate}
    \item La familia de subconjuntos de $X$
    \[
      \bar{\mathcal{F}}= \{A = B \cup E : B \in \mathcal{F} \text{ y } E \in N\}  
    \]
    es un $\sigma$-\'algebra sobre $X$ que contiene a $\mathcal{F}$.
    \item Si $B_1, B_2 \in \mathcal{F}$ y $E_1, E_2 \in N$ son tales que
    que $B_1 \cup E_1 = B_2 \cup E_2$, entonces $\mu(B_1) = \mu(B_2)$.
    \item Existe una \'unica medida $\bar{\mu}$ en el espacio medible
    $(X, \bar{\mathcal{F}})$ que extiende a $\mu$; es decir,
    $\bar{\mu}(A) = \mu(A)$ para todo $A \in \mathcal{F}$.
  \end{enumerate}
  Al espacio de medida $(X, \bar{\mathcal{F}}, \bar{\mu})$ se le
  denomina la compleci\'on de $(X, \mathcal{F}, \mu)$.
\end{statement}

\begin{proof}
\end{proof}

\begin{defn}[Colecci\'on $G_{\delta}$]
  Decimos que un conjunto $G \subset \BR^m$ esta en $G_{\delta}$ cuando $G$
  puede ser expresado como intersecci\'on numerable de conjuntos abiertos debe
  $\BR^m$.
\end{defn}

\begin{statement}{6}
  Sea $(\BR^m, \mathcal{L}, m)$ y $A \subset \BR^m$.
  Pruebe que las siguientes afirmaciones son equivalentes:
  \begin{enumerate}
    \item $A \in \mathcal{L}$.
    \item Para todo $\epsilon$ positivo podemos encontrar un abuerto $U$ tal
    que $A \subset U$ y $m(U \setminus A) < \epsilon$.
  \end{enumerate}
\end{statement}

\begin{proof}
\end{proof}

\begin{statement}{7}
  Pruebe que $(\BR^m, \mathcal{L}, m)$ es la compleción de
  $(\BR^m, \mathcal{B}_m, m |_{\mathcal{B}})$:
  \begin{enumerate}
    \item Pruebe que para todo $A \in \mathcal{L}$ existe $B \in G_{\delta}$
    tal que $m(A) = m(B)$.
    \item Concluya que $(\BR^m, \mathcal{L}, m)$ es la compleci\'on de
    $(\BR^m, \mathcal{B}_m, m |_{\mathcal{B}})$.
  \end{enumerate}
\end{statement}

\begin{proof}
\end{proof}

\begin{statement}{8}
  Sea $E \subset \BR^m$ un conjunto no vac\'io y $(A_n)_{n \in \BN}$ la
  sucesi\'on de abiertos
  \[
    A_n = \left\{x : d(x, E) < \frac{1}{n}\right\}.
  \]
  Demuestre lo siguiente:
  \begin{enumerate}
    \item Si $E$ es compacto, entonces $m(E) = \lim_{n \to \infty} m(A_n)$.
    \item La proposicion (1) no siempre es verdadera.
    % E cerrado mas no acotado
    % E acotado mas no cerrado
  \end{enumerate}
\end{statement}

\begin{proof}
\end{proof}

\begin{statement}{9}
  Sea $E \subset \BR^m$ Lesbesgue medible. Muestre que
  \[
    m(E) = \sup_{\substack{K \subset E \\ K \text{ es compacto}}}  m(K).
  \]
\end{statement}

\begin{proof}
\end{proof}

\begin{statement}{10}
  Sean $E_1$ y $E_2$ dos compactos en $\BR$ tales que $E_1 \subset E_2$.
  Sean $a$ y $b$ dos n\'umeros no negativos tales que
  $a = m(E_1)$ y $b = m(E_2)$. Demuestre que para todo $c \in (a, b)$ existe
  un compacto $E$ tal que $E_1 \subset E \subset E_2$ y $m(E) = c$.
\end{statement}

\begin{proof}
\end{proof}

\begin{statement}{11}
  Sea $X \subset \BR^m$ un conjunto cuya frontera $\partial X$ tiene
  medida de Lebesgue nula.
  \begin{enumerate}
    \item Pruebe que $X$ es Lebesgue medible.
    \item Pruebe que $m(X) = m(\text{int}(X)) = m(\bar{X})$.
  \end{enumerate}
\end{statement}

\begin{proof}
\end{proof}

\begin{statement}{12}
  Sea $X$ un conjunto y $\mathcal{F}$ una colecci\'on sobre $X$.
  Pruebe que las siguientes afirmaciones son equivalentes:
  \begin{enumerate}
    \item La colecci\'on $\mathcal{F}$ es un $\sigma$-\'algebra.
    \item La colecci\'on $\mathcal{F}$ es un $\pi$-sistema y un $\lambda$-sistema.
  \end{enumerate}
\end{statement}

\begin{proof}
\end{proof}

\begin{statement}{13}
  Sea $\epsilon$ un n\'umero positivo tal que $0 < \epsilon < 1$.
  \begin{enumerate}
    \item Construya un abierto denso $E \subset [0, 1]$ tal que $m(E) = \epsilon$.
    \item Construya un cerrado $F$ que no contenga ning\'un abierto no vac\'io
    tal que $m(F) = \epsilon$.
  \end{enumerate}
\end{statement}

\begin{proof}
\end{proof}

\begin{statement}{14}
  Sea $X$ un conjunto y $H$ una colecci\'on sobre $X$. Pruebe que las siguientes
  afirmaciones son equivalentes:
  \begin{enumerate}
    \item La colecci\'on $H$ es un $\lambda$-sistema.
    \item La colecci\'on $H$ satisface lo siguiente:
    \begin{enumerate}
      \item $X \in H$.
      \item Si $D \in H$, entonces $D^C \in H$.
      \item Si $(D_n)_{n \in \BN}$ est\'a en $H$ y es disjunta dos a dos,
      entonces $\cup_{n \in \BN} D_n \in H$.
    \end{enumerate}
  \end{enumerate}
\end{statement}

\begin{proof}
\end{proof}

\end{document}
