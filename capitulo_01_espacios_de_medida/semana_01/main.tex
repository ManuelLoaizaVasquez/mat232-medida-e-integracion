\documentclass{article}
\usepackage[utf8]{inputenc}
\usepackage{amsfonts,latexsym,amsthm,amssymb,amsmath,amscd,euscript}
\usepackage{mathtools}
\usepackage{framed}
% Descomentar fullpage cuando se quiera utilizar menos margen horizontal
%\usepackage{fullpage}
\usepackage{hyperref}
    \hypersetup{colorlinks=true,citecolor=blue,urlcolor =black,linkbordercolor={1 0 0}}

\newenvironment{statement}[1]{\smallskip\noindent\color[rgb]{1.00,0.00,0.50} {\bf #1.}}{}
\allowdisplaybreaks[1]

% Comandos para teoremas, definiciones, ejemplos, lemas, etc. para sus respectivos body types.
\renewcommand*{\proofname}{Prueba}
\renewcommand{\contentsname}{Contenido}

\newtheorem{theorem}{Teorema}
\newtheorem*{proposition}{Proposici\'on}
\newtheorem{lemma}[theorem]{Lema}
\newtheorem{corollary}[theorem]{Corolario}
\newtheorem{conjecture}[theorem]{Conjetura}
\newtheorem*{postulate}{Postulado}
\theoremstyle{definition}
\newtheorem{defn}[theorem]{Definici\'on}
\newtheorem{example}[theorem]{Ejemplo}

\theoremstyle{remark}
\newtheorem*{remark}{Observaci\'on}
\newtheorem*{notation}{Notaci\'on}
\newtheorem*{note}{Nota}

% Define tus comandos para hacer la vida más fácil.
\newcommand{\BR}{\mathbb R}
\newcommand{\BC}{\mathbb C}
\newcommand{\BF}{\mathbb F}
\newcommand{\BQ}{\mathbb Q}
\newcommand{\BZ}{\mathbb Z}
\newcommand{\BN}{\mathbb N}

\title{MAT232 Medida e Integraci\'on}
\author{Manuel Loaiza Vasquez}
\date{2021}

\begin{document}

\maketitle

\vspace*{-0.25in}
\centerline{Pontificia Universidad Cat\'olica del Per\'u}
\centerline{Lima, Per\'u}
\centerline{\href{mailto:manuel.loaiza@pucp.edu.pe}{{\tt manuel.loaiza@pucp.edu.pe}}}
\vspace*{0.15in}

\begin{framed}
  Esta es una plantilla en \LaTeX. Util\'izala en tus listas de ejercicios, notas de clases y mucho m\'as.
\end{framed}

\begin{statement}{1}
  Sea \{$X, \mathcal{F}$\} un espacio medible, pruebe que
  \begin{enumerate}
    \item Si $A, B \in \mathcal{F}$, entonces $A \setminus B \in \mathcal{F}$.
    \item Si $A, B \in \mathcal{F}$, entonces $A \Delta B \in \mathcal{F}$.
  \end{enumerate}
\end{statement}

\begin{proof}
\end{proof}

\begin{statement}{2}
  Sea $(X, \mathcal{F})$ un espacio medible y $E \subset X$, pruebe que
  \begin{enumerate}
    \item La colecci\'on $\{E \cap F : F \in \mathcal{F}\}$ es un sigma \'algebra sobre E.
    \item La colecci\'on $\sigma(f) = \{f^{-1}(A) : A \in \mathcal{F}\}$ es un sigma \'algebra sobre $Y$, donde $f: Y \to X$.
    \item El item (a) es consecuencia del item (b).
  \end{enumerate}
\end{statement}

\begin{proof}
\end{proof}

\begin{statement}{3}
  Sea $X$ un conjunto, pruebe lo siguiente
  \begin{enumerate}
    \item Si $(\mathcal{F}_{\lambda})_{\lambda \in L}$ es una familia arbitraria de sigma \'algebras sobre $X$, entonces $\cap_{\lambda \in L} \mathcal{F}$ es un sigma \'algebra sobre $X$.
    \item Si $\mathcal{C} \subset 2^X$, existe un sigma \'algebra $\mathcal{F}$ sobre $X$ tal que
      \begin{enumerate}
        \item $\mathcal{C} \subset \mathcal{F}$.
        \item Si $\mathcal{G}$ es un sigma \'algebra sobre $X$ tal que $\mathcal{C} \subset \mathcal{G}$, entonces $\mathcal{F} \subset \mathcal{G}$.
      \end{enumerate}
  \end{enumerate}
\end{statement}

\begin{proof}
\end{proof}

\begin{statement}{4}
  Sea $X$ un conjunto, pruebe lo siguiente
  \begin{enumerate}
    \item Si $A_1, A_2, \dots, A_n \subset X$ son disjuntos $2$ a $2$ y $\cup_{i = 1}^n A_i = X$, entonces $|\sigma(\{A_1, A_2, \dots, A_n\})| = 2^n$.
    \item Si $A_1, A_2, \dots, A_n \subset X$, entonces $\sigma(\{A_1, A_2, \dots, A_n\})$ es finito.
  \end{enumerate}
\end{statement}

\begin{proof}
\end{proof}

\begin{statement}{5}
  Si $\mathcal{F}$ es un sigma \'algebra infinito sobre $X$, entonces $\mathcal{F}$ no es numerable.
\end{statement}

\begin{proof}
\end{proof}

\begin{statement}{6}
  Sea $X$ un conjunto y $\mathcal{A} \subset 2^X$. Pruebe que $\sigma(\mathcal{A}) = \cup \{\sigma(\mathcal{C}) : \mathcal{C} \subset \mathcal{A} \text{ con } \mathcal{C} \text{ numerable}\}$.
\end{statement}

\begin{proof}
\end{proof}

\begin{statement}{7}
  Sea $(X, \mathcal{F}, \mu)$ un espacio de medida y considere $A, B, (A_n)_{n \in \BN}, (B_n)_{n \in \BN}$ en $\mathcal{F}$, pruebe las siguientes afirmaciones
  \begin{enumerate}
    \item Si $A \subset B$, entonces $\mu(A) \leq \mu(B)$.
    \item Si $A \subset B$ y $\mu(A) < \infty$, entonces $\mu(B \setminus A) = \mu(B) - \mu(A)$.
    \item $\mu(A \cup B) + \mu(A \cup B) = \mu(A) + \mu(B)$.
    \item Si $B_n \downarrow B$ y $\mu(B_1) < \infty$, entonces $\mu(B) = \inf_{n \in \BN} \mu(B_n)$.
    \item $\mu(\cup_{n \in \BN} A_n) \leq \sum_{n \in \BN} \mu(A_n)$.
  \end{enumerate}
\end{statement}

\begin{proof}
\end{proof}

\begin{statement}{8}
  Sea $(X, \mathcal{F}, \mu)$ un espacio medible y $(A_n)_{n \in \BN}$. Si $\sum_{n \in \BN} \mu(A_n) < \infty$, entonces $\mu(\cap_{n \in \BN} \cup_{k \geq n} A_k) = 0$.
\end{statement}

\begin{proof}
\end{proof}

\begin{statement}{9}
  Sea $(X, \mathcal{F})$ un espacio medible y $\mu: \mathcal{F} \to [0, \infty]$. Si $\mu$ es aditiva y $\sigma$-subaditiva, entonces $\mu$ es $\sigma$-aditiva.
\end{statement}

\begin{proof}
\end{proof}

\begin{statement}{10}
  Dados $(X, \mathcal{F})$, un espacio medible y una aplicaci\'on $\mu: \mathcal{F} \to [0, \infty]$, pruebe que las siguientes afirmaciones son equivalentes:
  \begin{enumerate}
    \item $(X, \mathcal{F}, \mu)$ es un espacio de medida.
    \item $\mu$ cumple lo siguiente:
      \begin{enumerate}
        \item $\mu(\emptyset) = 0$.
        \item Si $(A_n)_{n \in \BN}$ est\'a en $\mathcal{F}$ y son disjuntos dos a dos, entonces $\mu(\cup_{n \in \BN} A_n) = \sum_{n \in \BN} \mu(A_n)$.
      \end{enumerate}
  \end{enumerate}
\end{statement}

\begin{proof}
\end{proof}

\begin{statement}{11}
  Sea $(X, \mathcal{F})$ un espacio medible, pruebe lo siguiente:
  \begin{enumerate}
    \item Si $\mu$ y $\nu$ son medidas sobre $(X, \mathcal{F})$, entonces la aplicaci\'on $\rho: \mathcal{F} \to [0, \infty]$ dada por $\rho(A) = \mu(A) + \nu(A)$ tambi\'en es una medida sobre $(X, \mathcal{F})$.
    \item Si $\mu$ es una medidas sobre $(X, \mathcal{F})$ y $k > 0$, entonces la aplicaci\'on $\rho: \mathcal{F} \to [0, \infty]$ dada por $\rho(A) = k \mu(A)$ tambi\'en es una medida sobre $(X, \mathcal{F})$.
    \item Si $(\mu_n)_{n \in \BN}$ es una familia de medidas sobre $(X, \mathcal{F})$ y $(k_n)_{n \in \BN}$ es una sucesi\'on de reales positivos, entonces la aplicaci\'on $\rho: \mathcal{F} \to [0, \infty]$ dada por $\rho(A) = \sum_{n \in \BN} k_n \mu_n(A)$ tambi\'en es una medida sobre $(X, \mathcal{F})$.
  \end{enumerate}
\end{statement}

\begin{proof}
\end{proof}

\begin{statement}{12}
  Sea $(X, \mathcal{F}, \mu)$ un espacio de medida. Si $E \in \mathcal{F}$, entonces la aplicaci\'on dada por $\nu(A) = \mu(A \cap E)$ es una medida sobre $(X, \mathcal{F})$.
\end{statement}

\begin{proof}
\end{proof}

\begin{statement}{13}
  Sea $(\Omega, S, \mathbb{P})$ un espacio de probabilidad y $(A_n)_{n \in \BN}$. Si $\mathbb{P}(A_n) = 1$ para todo $n \in \BN$, entonces $\mathbb{P}(\cap_{n \in \BN} A_n) = 1$.
\end{statement}

\begin{proof}
\end{proof}
\end{document}
